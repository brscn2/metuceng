\pagenumbering{arabic}

\chapter{Introduction} \label{introduction}

\quad This document is written to describe the Software Architecture Description (SAD) for the open-source project FarmBot. The FarmBot is an open-source CNC (computer numerical control) farming project. It consists of a cartesian coordinate robot farming machine, software, and documentation, which includes a farming data repository.


\section{Purpose and objectives of FarmBot}

The project aims to create open-source and accessible technology for people to grow
food easily. In addition to that, the mission of the benefit corporation FarmBot Inc.
is to build a community that develops open-source hardware plans, software, data,
and documentation available to everyone to build their farming machine.
The system, therefore, paves the way for decentralization and democratization of food
production.

\section{Scope}

FarmBot is an open-source CNC farming project. The system aims to address the issues that have become prevalent with the ever-growing world population and its food demands. It significantly improves upon the conventional agriculture methods.
The scope of the system consists of the following:
\begin{itemize}
    \item Building a free and open database for farming and gardening knowledge
    \item A custom operating system, FarmBot OS, for maintaining connection and synchronization between the hardware and the web application using message brokers. Allowing scheduling of events, real-time control, and uploading various sensor data and logs
    \item A web app for easy configuration and control of the FarmBot, featuring real-time manual control capabilities, logging, drag-and-drop farm designing and a routine builder for FarmBot to execute scheduled routines
    \item Scalability for all sizes of operation from home-use to industrial-use.
    \item Big data acquisition and analysis for data-driven agriculture.
    \item Providing the user with various farm design options with space efficiency in mind
    \item Fully automated and optimized farming operations such as planting, watering, spraying, weed detection, and seed spacing with mono-crop and poly-crop capabilities
    \item Offering customizability to the user in adding new sensors, adjusting parameters of their FarmBot to their liking
\end{itemize}

\section{Stakeholders and their concerns}

The stakeholders of the system range from students, researchers, and robotic artwork creators to home users and people with disabilities. Characteristics of the stakeholders are explained below:
\begin{itemize}
    \item \textbf{Developers:} FarmBot being an open-source project, developers are concerned with improving existing features and implementing new ones, creating documentation for these features. Maintaining the technological ecosystem of FarmBot.

    \item \textbf{Students:} For students in any level of study from K12 (Kindergarten and grades 1-12) to university education, FarmBot provides a practical, fun and hands-on learning experience for STEM (Science, Technology, Engineering and Mathematics) learning objectives such as: \textbf{robotics}, \textbf{coding}, \textbf{soil science}, \textbf{biology}, and \textbf{much more}. They can be called novices in the domain of FarmBot. They do not have the necessary knowledge to comprehend the hardware-software interactions fully. They are naturally curious, which encourages them to come up with new features and ways to use the device. They want to have the freedom to use their creativity. In addition to that, the students do not have a long attention span, and it might be hard to keep them engaged.

    \item \textbf{Researchers:} For researchers, there are various ways to take advantage of the FarmBot. It offers repeatability, scale, speed, and low cost to researchers.
    \begin{itemize}
        \item \textbf{Repeatability}: The human error factor needs to be eliminated, leading to more accurate and repeatable results while experimenting. In addition, they want to schedule events (experiments) easily.
        \item \textbf{Scale}: Researchers want to easily test multiple groups of crops and run multiple experiments concurrently.
        \item \textbf{Speed}: Researchers want to create their own sequences and collect data automatically at any frequency and run experiments 24/7.
        \item \textbf{Cost}: Researchers want to avoid high labor costs
    \end{itemize}
    This class of stakeholders is highly educated and therefore can capitalize on FarmBot being highly customizable and integrable. They would like to create custom tooling (e.g., brush heads for coral farming) and custom regimens (e.g., speeding up coral growth) for their experiments. Collecting accurate data in a timely fashion is of utmost importance to them. Precise and continuous data flow from sensors and cameras frees the researchers from labor-intensive tasks and the potential for error. This class uses FarmBot for \textbf{phenotyping research}, \textbf{photogrammetry}, \textbf{off-world farming research} and such.

    \item \textbf{Robotic Artwork Creators:} This class of stakeholders approach to using FarmBot is different from the others. Their perspective is more creative and artistic. They use the various sensors and cameras of the device, feed the data from them into different algorithms such as \textbf{stable diffusion} to create art pieces. For this class, ability to customize farm designs and sequence customizing are crucial features. They have, similar to the researchers class, the required knowledge of technology and they are generally well educated.  

    \item \textbf{Home User:} The home user class is usually driven to use FarmBot because of emerging problems in the world such as climate change, environmental pollution due to plastic packaging and excessive pesticide use. This class takes pride in being able to grow its food itself and being able to grow challenging plants. They are not experts on the technology, they have general knowledge. They usually use their backyards, small gardens for farming. They generally do not have a lot of spare time to tend to their garden each day.

    \item \textbf{People with disabilites:} These people suffer from various physical and mental limitations in their day-to-day lives. They are not able to farm on their own using conventional farming techniques (eg. can't clean the garden from weeds). Centres for disabled people can use FarmBot for horticultural therapy (the art or practice of garden cultivation and management) or vocational training. This would help them integrate with the society and give them a sense of autonomy. They do not have the technical expertise due to physical and mental handicaps. 

    \item \textbf{Ministry of Agriculture:} This stakeholder is responsible for Agriculture activites and regulations within the country. As FarmBot is a complex agricultural system, the Ministry must audit it and make sure that FarmBot is not in breach of any agriculture regulations.
\end{itemize}



